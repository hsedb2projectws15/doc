\section{Mission Objectives}
\subsection{Import}
The aim is to support the import of data by parsing given files and validating the datatype of stored information.

Example: If a field should contain an integer but contains a string this should be marked as wrong, parsing should
resume at the next line.

\subsection{Cleansing}
Imported data should be marked as possibly incorrect if data in a field seems to be wrong.

Example: Student IDs should be exactly 6 digits long, anything else should be marked as wrong.

\subsection{Structuring}
Data is validated logically to make sure there are no redundancies.

Example: It is not necessary to store information about faculty in a table of departments, if personnel data is
stored in a separate table.

\subsection{Querying / Reporting}
The database can be queried for different kinds of information and is able to generate reports.

Example: The administrator can search for faculty, lecturers, rooms, times, departments and more.

\subsection{Update}
The database will accept updates if they are correct and do not violate any process rules. Otherwise the database
will not accept the data.

Example: If the administrator wants to put Prof. X at room Y at time Z, but has placed Prof. A there at the same
time, this query will yield an error.
