\section{Mission Objectives}
\subsection{Structuring}
Data will be stored in a manner that reduces the possibility of inconsistencies. The stored data
will contain information about lecturers, lectures, rooms and times. Possible restrictions
will be honored.

Example: If a restriction requires a specific room but the room is non-existent, this
will result in an error.

\subsection{Querying / Reporting}
The database can be queried for different kinds of information and is able to generate the reports
specified by the client.

Example: The administrator can search for faculty, lecturers, rooms, times, departments and more.

\subsection{Update}
The database will accept updates if they are correct and do not violate any process rules. Otherwise the database
will not accept the data.

Example: If a restriction puts a lecture at midnight, the restriction would violate the working hours of the
university and therefore will result in an error.

\subsection{Documentation}
A report of work planned for specific lecturers over a given time frame can be generated.

Example: The administrator wants to know what Prof X did in summer 2010.
